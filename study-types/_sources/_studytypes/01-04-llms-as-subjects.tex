% Reverted back to "subject", as participant seems to be primarily used for humans: see  https://dictionary.apa.org/subject and https://dictionary.apa.org/participant
\studytypesubsection{LLMs as Subjects (S4)}
\label{sec:llms-as-subjects}

\studytypeparagraph{Description}

In empirical studies, data is collected from participants through methods such as surveys, interviews, or controlled experiments.
LLMs can serve as virtual \emph{subjects} by simulating human behavior and interactions. If LLMs can generate responses that approximate those of human participants, they could be valuable for research involving user interactions, collaborative coding environments, and software usability assessments~\cite{ZHAO2025101167}.
To achieve this, prompt engineering techniques are widely employed; for instance, the \textit{Personas Pattern}~\cite{DBLP:journals/corr/abs-2308-07702} involves tailoring LLM responses to align with predefined profiles or roles that emulate specific user archetypes.
To serve as virtual subjects, generated responses should be indistinguishable from human-produced texts, consistent with the attitudes and sociodemographic information of the conditioning context (e.g., junior vs.\ senior developers), naturally aligned with the form, tone, and content of the simulated scenario, and reflect patterns in relationships between ideas, demographics, and behavior observed in comparable human data~\cite{DBLP:journals/corr/abs-2209-06899}.

\studytypeparagraph{Example(s)}

\citeauthor{DBLP:journals/ipm/XuSRGPLSH24}~\cite{DBLP:journals/ipm/XuSRGPLSH24} compiled a list of ways LLMs can support social science research, some of which transfer to empirical SE research. For example, LLMs can emulate human responses and behaviors in simulated interviews and focus groups~\citeauthor{DBLP:journals/ase/GerosaTSS24}~\cite{DBLP:journals/ase/GerosaTSS24}. Similarly, \citeauthor{bano2025doessoftwareengineerlook}~\cite{bano2025doessoftwareengineerlook}, investigated biases in LLM-generated candidate profiles in SE recruitment processes. They found biases favoring male candidates, lighter skin tones, and slim physiques, particularly for senior roles. LLMs may be able to simulate end-user feedback and behavior in usability studies, identify usability issues and offering suggestions for improvement based on predefined user personas.
