\studytypesubsection{LLMs as Judges}
\label{sec:llms-as-judges}

\studytypeparagraph{Description}

LLMs can act as judges or raters to evaluate properties of software artifacts.
For example, LLMs can be used to assess code readability, adherence to coding standards, or the quality of code comments.
Judgment is distinct from the more qualitative task of assigning a code or label to typically unstructured text (see Section \annotators).
It is also different from using LLMs for general software engineering tasks, as discussed in Section \llmsforresearcher.

\studytypeparagraph{Example(s)}

\citeauthor{DBLP:conf/re/LubosFTGMEL24}~\cite{DBLP:conf/re/LubosFTGMEL24} leveraged \href{https://www.llama.com/llama2/}{Llama-2} to evaluate the quality of software requirements statements. 
They prompted the LLM with the text below, where the words in curly brackets reflect the study parameters:

\judgesexample

\citeauthor{DBLP:conf/re/LubosFTGMEL24} evaluated the LLM's output against human judges, to assess how well the LLM matched experts. 
Agreement can be measured in many ways; this study used Cohen's kappa~\cite{cohen60} and found moderate agreement for simple requirements and poor agreement for more complex requirements.
However, human evaluation of machine judgments may not be the same as evaluation of human judgments and is an open area of research~\cite{DBLP:journals/corr/abs-2410-03775}. 
A crucial decision in studies focusing on LLMs as judges is the number of examples provided to the LLM.
This might involve no tuning (zero-shot), several examples (few-shot), or providing many examples (closer to traditional training data).
In the example above, \citeauthor{DBLP:conf/re/LubosFTGMEL24} chose zero-shot tuning, providing no specific guidance besides the provided context; they did not show the LLM what a `yes' or `no' answer might look like. 
In contrast, \citeauthor{wang2025multimodalrequirementsdatabasedacceptance}~\cite{wang2025multimodalrequirementsdatabasedacceptance} provided a rubric to an LLM when treating LLMs as a judge to produce interpretable scales (0 to 4) for the overall quality of acceptance criteria generated from user stories.
They combined this with a lower-level reward stage to refine acceptance criteria, which significantly improved correctness, clarity, and alignment with the user stories.

\studytypeparagraph{Advantages}

Depending on the model configuration, LLMs can provide relatively consistent evaluations.
They can further help mitigate human biases and the general variability that human judges might introduce.
This may lead to more reliable and reproducible results in empirical studies, to the extent that these models can be reproduced or checkpointed.
LLMs can be much more efficient and scale more easily than the equivalent human approach.
With LLM automation, entire datasets can be assessed, as opposed to subsets, and the assessment produced can be at the selected level of granularity, e.g., binary outputs (`yes' or `no') or defined levels (e.g., 0 to 4).
However, the main constraint that varies by model and budget is the input context size, i.e., the number of tokens one can pass into a model.
For example, the upper bound of the context that can be passed to OpenAi's \textsf{o1-mini} model is \href{https://help.openai.com/en/articles/9855712-openai-o1-models-faq-chatgpt-enterprise-and-edu}{32k tokens}.

\studytypeparagraph{Challenges}

When relying on the judgment of LLMs, researchers must build a \textit{reliable} process for generating judgment labels that considers the non-deterministic nature of LLMs and report the intricacies of that process transparently~\cite{DBLP:journals/corr/abs-2412-12509}.
For example, the order of options has been shown to affect LLM outputs in multiple-choice settings~\cite{DBLP:conf/naacl/PezeshkpourH24}. 
In addition to reliability, other quality attributes to consider include the \textit{accuracy} of the labels. % and the speed and \textit{scalability} of the process. 
For example, a reliable LLM might be reliably inaccurate and wrong. 
Evaluating and judging large numbers of items, for example, to perform fault localization on the thousands of bugs that large open-source projects have to deal with, comes with costs in clock time, compute time, and environmental sustainability.
Evidence shows that LLMs can behave differently when reviewing their own outputs~\cite{NEURIPS2024_7f1f0218}.
In more human-oriented datasets (such as discussions of pull requests) LLMs may suffer from well-documented biases and issues with fairness~\cite{Gallegos2024BiasAF}. 
For tasks in which human judges disagree significantly, it is not clear if an LLM judge should reflect the majority opinion or act as an independent judge.
The underlying statistical framework of an LLM usually pushes outputs towards the most likely (majority) answer. 
There is ongoing research on how suitable LLMs are as independent judges.
Questions about bias, accuracy, and trust remain~\cite{DBLP:journals/corr/abs-2406-18403}.
There is reason for concern about LLMs judging student assignments or doing peer review of scientific papers~\cite{DBLP:conf/coling/ZhouC024}.
Even beyond questions of technical capacity, ethical questions remain, particularly if there is some implicit expectation that a human is judging the output.
Involving a human in the judgment loop, for example, to contextualize the scoring, is one approach~\cite{panHumanCenteredDesignRecommendations2024}. 
However, the lack of large-scale ground truth datasets for benchmarking LLM performance in judgment studies hinders progress in this area.
