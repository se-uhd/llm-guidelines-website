\studytypesubsection{Studying LLM Usage in Software Engineering (S5)}
\label{sec:studying-llm-usage-in-software-engineering}

\studytypeparagraph{Description}

Studying how software engineers use LLMs and LLM-based tools is important for understanding the current state of practice in SE.
Researchers can observe software engineers' usage of LLM-based tools in the field, or study if and how they adopt such tools, their usage patterns, as well as perceived benefits and challenges.
Surveys, interviews, observational studies, or analysis of usage logs can provide insights into how LLMs are integrated into development processes, how they influence decision making, and what factors affect their acceptance and effectiveness. 
Such studies can inform improvements for existing LLM-based tools, motivate the design of novel tools, or derive best practices for LLM-assisted software engineering.
They can also uncover risks or deficiencies of existing tools.

\studytypeparagraph{Example(s)}

Based on a convergent mixed-methods study, \citeauthor{russo2024navigating} has found that early adoption of generative AI by software engineers is primarily driven by compatibility with existing workflows~\cite{russo2024navigating}.
\citeauthor{DBLP:journals/pacmse/KhojahM0N24} investigated the use of ChatGPT (GPT-3.5) by professional software engineers in a week-long observational study~\cite{DBLP:journals/pacmse/KhojahM0N24}.
They found that most developers do not use the code generated by ChatGPT directly but instead use the output as a guide to implement their own solutions.
%Furthermore, they recommend that future research investigate the use of ChatGPT in non code-related SE tasks and for training and learning SE concepts.
\citeauthor{DBLP:conf/csee/AzanzaPIG24}'s case study found that LLMs could \enq{enhance personalized, instant onboarding support; however, relying on proprietary external LLMs poses significant data privacy risks}~\cite{DBLP:conf/csee/AzanzaPIG24}. 
 \citeauthor{DBLP:conf/icsa/JahicS24} found that most participants from 15 software companies they surveyed had already adopted AI (especially ChatGPT) for SE tasks; but cited copyright and privacy issues, as well as inconsistent or low-quality outputs, as barriers to adoption~\cite{DBLP:conf/icsa/JahicS24}. 
Retrospective studies that analyze data generated while developers use LLMs can provide additional insights into human-LLM interactions.
For example, researchers can employ data mining methods to build large-scale conversation datasets, such as the DevGPT dataset introduced by \citeauthor{DBLP:conf/msr/XiaoTHM24}~\cite{DBLP:conf/msr/XiaoTHM24}.
Conversations can then be analyzed using quantitative~\cite{DBLP:conf/msr/RabbiCZI24} and qualitative~\cite{DBLP:conf/msr/MohamedPP24} analysis methods.
