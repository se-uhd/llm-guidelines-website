\studytypesubsection{LLMs for New Software Engineering Tools}
\label{sec:llms-for-new-software-engineering-tools}

\studytypeparagraph{Description}

LLMs are being integrated into new tools that support software engineers in their daily tasks, e.g., to assist in code comprehension~\cite{DBLP:conf/chi/YanHWH24} and test case generation~\cite{DBLP:journals/tse/SchaferNET24}.
One way of integrating LLM-based tools into software engineers' workflows are GenAI agents.
Unlike traditional LLM-based tools, these agents are capable of acting autonomously and proactively, are often tailored to meet specific user needs, and can interact with external environments~\cite{takerngsaksiri2024human,wiesinger2025agents}.
From an architectural perspective, GenAI agents can be implemented in various ways~\cite{wiesinger2025agents}.
However, they generally share three key components: (1) a reasoning mechanism that guides the LLM (often enabled by advanced prompt engineering), (2) a set of tools to interact with external systems (e.g., APIs or databases), and (3) a user communication interface that extends beyond traditional chat-based interactions~\cite{DBLP:conf/icsm/RichardsW24, DBLP:journals/tmlr/SumersYN024, DBLP:journals/corr/abs-2309-07870}.
Researchers can also test and compare different tool architectures to increase artifact quality and developer satisfaction.

\studytypeparagraph{Example(s)}

\citeauthor{DBLP:conf/chi/YanHWH24} proposed IVIE, a tool integrated into the VS Code graphical interface that generates and explains code using LLMs~\cite{DBLP:conf/chi/YanHWH24}.
The authors focused more on the presentation, providing a user-friendly interface to interact with the LLM. 
\citeauthor{DBLP:journals/tse/SchaferNET24}~\cite{DBLP:journals/tse/SchaferNET24} presented a large-scale empirical evaluation on the effectiveness of LLMs for automated unit test generation.
They presented TestPilot, a tool that implements an approach in which the LLM is provided with prompts that include the signature and implementation of a function under test, along with usage examples extracted from the documentation.
\citeauthor{DBLP:conf/icsm/RichardsW24} introduced a preliminary GenAI agent designed to assist developers in understanding source code by incorporating a reasoning component grounded in the theory of mind~\cite{DBLP:conf/icsm/RichardsW24}.

\studytypeparagraph{Advantages}

From an engineering perspective, developing LLM-based tools is easier than implementing many traditional SE approaches such as static analysis or symbolic execution.
Depending on the capabilities of the underlying model, it is also easier to build tools that are independent of a specific programming language.
This enables researchers to build tools for a more diverse set of tasks.
In addition, it allows them to test their tools in a wider range of contexts.

\studytypeparagraph{Challenges}

Traditional approaches, such as static analysis, are deterministic. LLMs are not.
Although the non-determinism of LLMs can be mitigated using configuration parameters and prompting strategies, this poses a major challenge.
It can be challenging for researchers to evaluate the effectiveness of a tool, as minor changes in the input can lead to major differences in the performance.
Since the exact training data is often not published by model vendors, a reliable assessment of tool performance for unknown data difficult.
From an engineering perspective, while open models are available, the most capable ones require substantial hardware resources.
Using cloud-based APIs or relying on third-party providers for hosting, while seemingly a potential solution, introduces new concerns related to data privacy and security.
