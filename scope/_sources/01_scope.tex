\documentclass[pdftex,11pt,a4paper]{article}

% language and encoding
\usepackage[utf8]{inputenc} % set document encoding to UTF-8
\usepackage[main=english, ngerman]{babel} % adjust language depending on thesis language (affects hyphenation)
\usepackage[T1]{fontenc} % set font encoding so that special characters are displayed correctly

% layout and formatting
\usepackage[paper=a4paper, inner=30mm, outer=25mm, top=30mm, bottom=25mm]{geometry}  % set page margins
\usepackage[parfill]{parskip} % use newlines for paragraphs (more similar to Markdown)
\usepackage{xspace}  % control whitespaces
\usepackage[hyphens]{url} % line breaks in URLs
\usepackage{hyperref}
\usepackage{csquotes}% environment for quotes
\usepackage[super]{nth}
\newcommand*{\enq}[1]{\enquote{{\itshape#1}}} % use italics font for quotes

% math formulas
\usepackage{amsmath}
\usepackage{amssymb} 
\usepackage{eucal} % more curly versions for \mathcal{...}
\usepackage{nicefrac} % nicer fractions
\usepackage{bm} % bold math symbols

% citations
\usepackage[square,numbers,sort&compress]{natbib}

% editorial commands
\newcommand{\todo}[1]{{\textbf{TODO:}\ \textit{#1}}} % command for TODOs
\newcommand{\comment}[1]{{\textbf{Comment:}\ \textit{#1}}} % command for review comments

% RFC 2119 (https://www.rfc-editor.org/rfc/rfc2119)
% MUST: absolute requirement
\newcommand{\must}{\textbf{MUST}\xspace}
% MUST NOT: absolute prohibition
\newcommand{\mustnot}{\textbf{MUST NOT}\xspace}
% SHOULD: there may exist valid reasons in particular circumstances to ignore a  particular item, but the full implications must be understood and carefully weighed before choosing a different course
\newcommand{\should}{\textbf{SHOULD}\xspace}
% SHOULD NOT: there may exist valid reasons in particular circumstances when the particular behavior is acceptable or even useful, but the full implications should be understood and the case carefully weighed before implementing any behavior described with this label
\newcommand{\shouldnot}{\textbf{SHOULD NOT}\xspace}
% MAY: an item is truly optional
\newcommand{\may}{\textbf{MAY}\xspace}

% command to indicate where certain information should be reported
\newcommand{\paper}{PAPER\xspace}
\newcommand{\supplementarymaterial}{SUPPLEMENTARY MATERIAL\xspace}

% configure enumerate/itemize
\usepackage[inline]{enumitem}

% commands to reference sections

% scope
\newcommand{\scope}{\href{/scope/}{Motivation and Scope}\xspace}

% study types
\newcommand{\studytypes}{\href{/study-types}{Study Types}\xspace}
\newcommand{\llmsforresearcher}{\href{/study-types/\#introduction-llms-as-tools-for-software-engineering-researchers}{LLMs as Tools for Software Engineering Researchers}\xspace}
\newcommand{\annotators}{\href{/study-types\#llms-as-annotators}{LLMs as Annotators}\xspace}
\newcommand{\judges}{\href{/study-types\#llms-as-judges}{LLMs as Judges}\xspace}
\newcommand{\synthesis}{\href{/study-types\#llms-for-synthesis}{LLMs for Synthesis}\xspace}
\newcommand{\subjects}{\href{/study-types\#llms-as-subjects}{LLMs as Subjects}\xspace}
\newcommand{\llmsforengineers}{\href{/study-types/\#introduction-llms-as-tools-for-software-engineers}{LLMs as Tools for Software Engineers}\xspace}
\newcommand{\llmusage}{\href{/study-types\#studying-llm-usage-in-software-engineering}{Studying LLM Usage in Software Engineering}\xspace}
\newcommand{\newtools}{\href{/study-types\#llms-for-new-software-engineering-tools}{LLMs for New Software Engineering Tools}\xspace}
\newcommand{\benchmarkingtasks}{\href{/study-types\#benchmarking-llms-for-software-engineering-tasks}{Benchmarking LLMs for Software Engineering Tasks}\xspace}

%guidelines
\newcommand{\guidelines}{\href{/guidelines}{Guidelines}\xspace}
\newcommand{\usagerole}{\href{/guidelines\#declare-llm-usage-and-role}{Declare LLM Usage and Role}\xspace}
\newcommand{\modelversion}{\href{/guidelines\#report-model-version-configuration-and-customizations}{Report Model Version, Configuration, and Customizations}\xspace}
\newcommand{\toolarchitecture}{\href{/guidelines\#report-tool-architecture-beyond-models}{Report Tool Architecture beyond Models}\xspace}
\newcommand{\humanvalidation}{\href{/guidelines\#use-human-validation-for-llm-outputs}{Use Human Validation for LLM Outputs}\xspace}
\newcommand{\prompts}{\href{/guidelines\#report-prompts-their-development-and-interaction-logs}{Report Prompts, their Development, and Interaction Logs}\xspace}
\newcommand{\openllm}{\href{/guidelines\#use-an-open-llm-as-a-baseline}{Use an Open LLM as a Baseline}\xspace}
\newcommand{\benchmarksmetrics}{\href{/guidelines\#use-suitable-baselines-benchmarks-and-metrics}{Use Suitable Baselines, Benchmarks, and Metrics}\xspace}
\newcommand{\limitationsmitigations}{\href{/guidelines\#report-limitations-and-mitigations}{Report Limitations and Mitigations}\xspace}

% custom format for subsubsections and paragraphs
\newcommand{\guidelinesubsubsection}[1]{\subsubsection{#1}}
\newcommand{\studytypesubsection}[1]{\subsection{#1}}
\newcommand{\studytypeparagraph}[1]{\subsubsection{#1}}
\newcommand{\scopeparagraph}[1]{\subsubsection{#1}}

% adjust layout depending on document type
\newcommand{\modelversionsummary}{In summary, our recommendation is to report:

\textbf{(1)} Model/tool name and version (\must in \paper);
\textbf{(2)} All relevant configured parameters that affect output generation (\must in \paper);
\textbf{(3)} Default values of all available parameters (\should);
\textbf{(4)} Checksum/fingerprint of used model version and configuration (\may);
\textbf{(5)} Additional properties such as context window size (\may).

For fine-tuned models, additional recommendations apply:

\textbf{(1)} Fine-tuning goal (\must in \paper);
\textbf{(2)} Fine-tuning dataset creation and characterization (\must in \paper);
\textbf{(3)} Fine-tuning parameters and procedure (\must in \paper);
\textbf{(4)} Fine-tuning dataset and fine-tuned model weights (\should);
\textbf{(5)} Validation metrics and benchmarks (\should).
}
\newcommand{\judgesexample}{\begin{quote}
\small
Your task is to evaluate the quality of a software requirement.\\
Evaluate whether the following requirement is \{quality\_characteristic\}. \\
\{quality\_ characteristic\} means: \{quality\_characteristic\_explanation\}\\
The evaluation result must be: `yes' or `no'.\\
Request: Based on the following description of the project:
\{project\_description\}\\
Evaluate the quality of the following requirement: \{requirement\}.\\
Explain your decision and suggest an improved version.\\
\end{quote}
}


\begin{document}

\section{Motivation and Scope}

\subsection{Motivation}

In the short period since the release of ChatGPT in November 2022, large language models (LLMs) have changed the software engineering (SE) research landscape.
Although there are numerous opportunities to use LLMs to support SE research and development tasks, solid science needs rigorous empirical evaluations to explore the effectiveness, performance, and robustness of using LLMs to automate different research and development tasks.
LLMs can, for example, be used to support literature reviews, reduce development time, and generate software documentation.
However, it is often unclear how valid, reproducible, and replicable empirical studies involving LLM are.
This uncertainty poses significant challenges for researchers and practitioners who seek to draw reliable conclusions from empirical studies.

The importance of open science practices and documenting the study setup is not to be underestimated~\cite{DBLP:journals/corr/abs-2412-17859}.
One of the primary risks in creating irreproducible and irreplicable results based on studies involving LLMs stems from the variability in model performance due to their inherent non-determinism, but also due to differences in configuration, training data, model architecture, or evaluation metrics.
Slight changes can lead to significantly different results.
The lack of standardized benchmarks and evaluation protocols further hinders the reproducibility and replicability of study results.
Without detailed information on the exact study setup, benchmarks, and metrics used, it is challenging for other researchers to replicate results using different models and tools.
These issues highlight the need for clear guidelines and best practices for designing and reporting studies involving LLMs.
While the SE research community has developed guidelines for conducting and reporting specific types of empirical studies such as controlled experiments (e.g., \emph{Experimentation in Software Engineering}~\cite{DBLP:books/sp/WohlinRHORW24}, \emph{Guide to Advanced Empirical Software Engineering}~\cite{DBLP:books/sp/08/SSS2008}) or their replications (e.g., \emph{A Procedure and Guidelines for Analyzing Groups of Software Engineering Replications}~\cite{DBLP:journals/tse/SantosVOJ21}),  we believe that LLMs have specific intrinsic characteristics that require specific guidelines for researchers to achieve an acceptable level of reproducibility and replicability (see also our previous position paper~\cite{DBLP:conf/wsese/0001BFB25}).
For example, even if we knew the specific version of a commercial LLM used for an empirical study, the reported task performance could still change over time, since commercial models are known to evolve beyond version identifiers~\cite{DBLP:journals/corr/abs-2307-09009}.
Moreover, commercial providers do not guarantee the availability of old model or tool versions indefinitely.
In addition to version differences, LLM performance varies widely depending on configured parameters such as temperature.
Therefore, not reporting the parameter settings severely impacts reproducibility.
Even for ``open'' models such as \emph{Llama}, we do not know how they were fine-tuned for specific tasks and what the exact training data was~\cite{Gibney2024}.
A general problem when evaluating LLMs' performance is that we do not know whether the solution to a certain problem was part of the training data or not.

So far, there are no holistic guidelines for conducting and reporting studies involving LLMs in SE research.
With this community effort, we try to fill this gap.
After outlining our \scope, we continue by introducing a taxonomy of \studytypes before presenting eight \guidelines that the authors of this article co-developed.
The most recent version is always available online.\footnote{\url{https://llm-guidelines.org}}
Other researchers can suggest changes via a public GitHub repository.\footnote{\url{https://github.com/se-uhd/llm-guidelines-website}}

\subsection{Scope}
\label{sec:scope}

First, we want to clarify that our focus is on LLMs, that is, natural language use cases.
Multi-modal foundational models are beyond the scope of our study types and guidelines.
We are aware that these foundational models have great potential to support software engineering research and practice.
However, due to the diversity of artifacts that can be generated or used as input (e.g., images, audio, and video) and the more demanding hardware requirements, we deliberately focus on LLMs only.
However, our guidelines could be extended in the future to include foundational models beyond natural language text.

Second, given the exponential growth in LLM usage across all research domains, we also want to define the research contexts in which our guidelines apply.
LLMs are already widely used to support several aspects of the overall research process, from fairly simple tasks such as proof-reading, spell-checking, and text translation, to more complex activities such as data coding and synthesis of literature reviews.
The \studytypes and \guidelines we describe are tailored to software engineering (SE) research, but we expect many of our study types to generalize beyond that domain.
On the practical side, we focus on AI for software engineering (AI4SE), that is, studying the support and automation of SE tasks with the help of artificial intelligence (AI), more specifically LLMs (see Section~\llmsforengineers).
In terms of research support, we focus on empirical SE research supported by LLMs (see Section~\llmsforresearcher).
By research support, we mean the active involvement of LLMs in data collection, processing, or analysis.
We consider LLMs supporting the study design or the writing process to be out of scope.

Third, our guidelines mainly target researchers planning, designing, or conducting empirical studies involving LLMs.
Although researchers who review scientific articles written by others can also use our guidelines, for example, to check whether the authors adhere to the essential \must requirements, reviewers are not our main target audience.

\subsection{References}

\bibliographystyle{plain}
\bibliography{../../literature.bib}

\end{document}
