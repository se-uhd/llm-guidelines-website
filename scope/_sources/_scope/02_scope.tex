\scopeparagraph{Software Engineering as our Target Discipline}

We target SE research because existing cross-disciplinary guidelines do not address its needs. For instance, Gallifant et al.'s~\cite{Gallifant2025} guidelines for LLM use in healthcare research include discipline-specific items irrelevant to most SE studies (e.g., clinical-care usability) while lacking guidance on tool architectures, benchmarking, and code-specific evaluation that SE research requires.
Moreover, SE research employs a wide variety of empirical methods~\cite{ralph2021empiricalstandardssoftwareengineering, DBLP:books/sp/WohlinRHORW24}. Our work builds on \citeauthor{sallou2024breaking}'s vision paper on mitigating validity threats in LLM-based SE research~\cite{sallou2024breaking} and our own position paper~\cite{DBLP:conf/wsese/0001BFB25}, providing more detailed advice for a wider variety of research methods organized into a taxonomy of \studytypes (Section~\ref{sec:study-types}).

\scopeparagraph{Focus on Text-Based Use Cases}

While multi-modal foundation models that use or generate images, audio, or video may also support SE research and practice, we focus on textual use cases of LLMs (e.g., in natural language or programming languages). Many of our guidelines---particularly those concerning model reporting, prompt documentation, and reproducibility---are likely applicable to multi-modal settings as well, but we leave their explicit validation to future work.

\scopeparagraph{Focus on Direct Development or Research Support}

While researchers may use LLMs for many peripheral tasks (e.g., proof-reading, spell-checking, translation), our guidelines focus on their direct role in empirical research and engineering practice.
For engineers, we focus on the use of LLMs to automate SE tasks, that is, artificial intelligence (AI) for software engineering (AI4SE) (see~\llmsforengineers).
This includes agentic systems that autonomously plan and execute multi-step tasks using LLMs (see~\toolarchitecture).
For researchers, we focus on the use of LLMs to automate empirical research tasks such as data collection, processing, or analysis (see~\llmsforresearcher).

\scopeparagraph{Researchers as our Target Audience}

Our guidelines are intended to help SE researchers design, plan, conduct,
and report empirical studies involving LLMs, and to support scholarly peer
review of such studies. Each guideline includes an \emph{Advice for
Reviewers} subsection with targeted assessment suggestions.
Our guidelines focus on what to report and how; they complement but do not replace methodological guidance for designing specific types of empirical studies.

\scopeparagraph{How to Navigate this Paper}

This paper is structured to support different reading strategies depending on the reader's goal.
\emph{Researchers planning a new study} may start with the taxonomy of
study types (see \studytypes) to identify which types apply
to their planned work, then consult Table~\ref{tab:guideline-matrix} to
determine which guidelines are requirements (\must) and which are
recommendations (\should) for those study types. Each guideline section
opens with a \tldr summary in a shaded box, allowing
readers to quickly assess relevance before reading the full text.
\emph{Researchers writing up results} may prefer to start with the
checklist in Appendix~\ref{sec:checklist}, which organizes actionable items
by typical paper sections (Introduction, Research Design and Methods, Results, etc.).
Table~\ref{tab:guidelines} provides a quick reference to all eight
guidelines, and Table~\ref{tab:rationale-recommendations} maps each
guideline's rationale to its key recommendations.
\emph{Reviewers} can use the \emph{Advice for Reviewers} subsection at the
end of each guideline for targeted guidance on assessing manuscripts.
Table~\ref{tab:guideline-matrix} helps reviewers identify which guidelines
apply to the study type under review.