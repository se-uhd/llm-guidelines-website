First, we want to clarify that our focus is on LLMs, that is, natural language use cases.
Multi-modal foundational models are beyond the scope of our study types and guidelines.
We are aware that these foundational models have great potential to support software engineering research and practice.
However, due to the diversity of artifacts that can be generated or used as input (e.g., images, audio, and video) and the more demanding hardware requirements, we deliberately focus on LLMs only.
However, our guidelines could be extended in the future to include foundational models beyond natural language text.

Second, given the exponential growth in LLM usage across all research domains, we also want to define the research contexts in which our guidelines apply.
LLMs are already widely used to support several aspects of the overall research process, from fairly simple tasks such as proof-reading, spell-checking, and text translation, to more complex activities such as data coding and synthesis of literature reviews.
The \studytypes and \guidelines we describe are tailored to software engineering (SE) research, but we expect many of our study types to generalize beyond that domain.
On the practical side, we focus on AI for software engineering (AI4SE), that is, studying the support and automation of SE tasks with the help of artificial intelligence (AI), more specifically LLMs (see Section~\llmsforengineers).
In terms of research support, we focus on empirical SE research supported by LLMs (see Section~\llmsforresearcher).
By research support, we mean the active involvement of LLMs in data collection, processing, or analysis.
We consider LLMs supporting the study design or the writing process to be out of scope.

Third, our guidelines mainly target researchers planning, designing, or conducting empirical studies involving LLMs.
Although researchers who review scientific articles written by others can also use our guidelines, for example, to check whether the authors adhere to the essential \must requirements, reviewers are not our main target audience.
