In this section, we outline the scope of the study types and guidelines we present.

\scopeparagraph{SE as our Target Discipline}

We target \emph{software engineering} (SE) research conducted in academic or industry contexts.
While other disciplines have started developing community guidelines for reporting empirical studies involving LLMs, SE has not seen a holistic effort in that direction.
In healthcare, for example, \citeauthor{Gallifant2025} presented the \emph{TRIPOD-LLM} guidelines~\cite{Gallifant2025}, which are centered around a checklist for \enq{good reporting of studies that are developing, tuning, prompt engineering or evaluating an LLM}~\cite{Gallifant2025}.
Although the checklist overlaps with our \guidelines~(e.g., reporting the LLM name and version, comparing performance between LLMs, humans and other benchmarks), they are healthcare-specific (e.g., they require to \enq{explain the healthcare context} and \enq{therapeutic and clinical workflow}).
Instead of a checklist, we distinguish \must and \should criteria and provide brief \emph{tl;dr} summaries highlighting our most important recommendations.
SE research utilizes a broader range of empirical methods than many other disciplines.
Therefore, we are convinced that SE-specific guidelines need to be developed in combination with a \studytypes taxonomy such as ours.
In SE, \citeauthor{sallou2024breaking} have presented a vision paper on mitigating validity threats in LLM-based SE research.
Although their guidelines partially overlap with ours (e.g., repeating experiments to handle output variability, considering data leakage), our guidelines cover a broader range of study types and provide more detailed advise.
For example, we contextualize our guidelines by study type, distinguish LLMs and LLM-based tools, talk about models, baselines, benchmarks, and metrics, and explicitly address human validation.
Moreover, our guidelines are the result of an extensive coordination process between a large number of experts in empirical SE, hence a first step towards true community guidelines.



\scopeparagraph{Focus on Natural Language Use Cases}

We want to clarify that our focus is on LLMs, that is, natural language use cases.
Multi-modal foundational models are beyond the scope of our study types and guidelines.
We are aware that these foundational models have great potential to support SE research and practice.
However, due to the diversity of artifacts that can be generated or used as input (e.g., images, audio, and video) and the more demanding hardware requirements, we deliberately focus on LLMs only.
However, our guidelines could be extended in the future to include foundational models beyond natural language text.

\scopeparagraph{Focus on Direct Tool or Research Support}

Given the exponential growth in LLM usage across all research domains, we also want to define the research contexts in which our guidelines apply.
LLMs are already widely used to support several aspects of the overall research process, from fairly simple tasks such as proof-reading, spell-checking, and text translation, to more complex activities such as data coding and synthesis of literature reviews.
Regarding tools, we focus on use cases in the area of AI for software engineering (AI4SE), that is, studying the support and automation of SE tasks with the help of artificial intelligence (AI), more specifically LLMs (see Section~\llmsforengineers).
For research support, we focus on empirical SE research supported by LLMs (see Section~\llmsforresearcher).
By research support, we mean the active involvement of LLMs in data collection, processing, or analysis.
We consider LLMs supporting the study design or the writing process to be out of scope.

\scopeparagraph{Researchers as our Target Audience}

Third, our guidelines mainly target SE researchers planning, designing, conducting, and reporting empirical studies involving LLMs.
Although researchers who review scientific articles written by others can also use our guidelines, for example, to check whether the authors adhere to the essential \must requirements we present, reviewers are not our main target audience.
