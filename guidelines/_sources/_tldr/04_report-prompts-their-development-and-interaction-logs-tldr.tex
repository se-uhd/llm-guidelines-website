Researchers \must publish all prompts, including their structure, content, formatting, and dynamic components. If full prompt disclosure is not feasible, for example, due to privacy or confidentiality concerns, summaries or examples \should be provided. Prompt development strategies (e.g., zero-shot, few-shot), rationale, and selection process \must be described. When prompts are long or complex, input handling and token optimization strategies \must be documented. For dynamically generated or user-authored prompts, generation and collection processes \must be reported. Prompt reuse across models and configurations \must be specified. Researchers \should report prompt revisions and pilot testing insights. For agentic systems, developed plans \should be reported and interaction logs \should be generalized to include human-agent interaction traces. For studies using AI coding agents, all context files used to configure agent behavior \must be reported as part of the \supplementarymaterial. To address model non-determinism and ensure reproducibility, especially when targeting SaaS-based commercial tools, full interaction logs (prompts and responses) \should be included if privacy and confidentiality can be ensured.