\documentclass[pdftex,11pt,a4paper]{article}

% language and encoding
\usepackage[utf8]{inputenc} % set document encoding to UTF-8
\usepackage[main=english, ngerman]{babel} % adjust language depending on thesis language (affects hyphenation)
\usepackage[T1]{fontenc} % set font encoding so that special characters are displayed correctly

% layout and formatting
\usepackage[paper=a4paper, inner=30mm, outer=25mm, top=30mm, bottom=25mm]{geometry}  % set page margins
\usepackage[parfill]{parskip} % use newlines for paragraphs (more similar to Markdown)
\usepackage{xspace}  % control whitespaces
\usepackage[hyphens]{url} % line breaks in URLs
\usepackage{hyperref}
\usepackage{csquotes}% environment for quotes
\usepackage[super]{nth}
\newcommand*{\enq}[1]{\enquote{{\itshape#1}}} % use italics font for quotes

% math formulas
\usepackage{amsmath}
\usepackage{amssymb} 
\usepackage{eucal} % more curly versions for \mathcal{...}
\usepackage{nicefrac} % nicer fractions
\usepackage{bm} % bold math symbols

% citations
\usepackage[square,numbers,sort&compress]{natbib}

% editorial commands
\newcommand{\todo}[1]{{\textbf{TODO:}\ \textit{#1}}} % command for TODOs
\newcommand{\comment}[1]{{\textbf{Comment:}\ \textit{#1}}} % command for review comments

% RFC 2119 (https://www.rfc-editor.org/rfc/rfc2119)
% MUST: absolute requirement
\newcommand{\must}{\textbf{MUST}\xspace}
% MUST NOT: absolute prohibition
\newcommand{\mustnot}{\textbf{MUST NOT}\xspace}
% SHOULD: there may exist valid reasons in particular circumstances to ignore a  particular item, but the full implications must be understood and carefully weighed before choosing a different course
\newcommand{\should}{\textbf{SHOULD}\xspace}
% SHOULD NOT: there may exist valid reasons in particular circumstances when the particular behavior is acceptable or even useful, but the full implications should be understood and the case carefully weighed before implementing any behavior described with this label
\newcommand{\shouldnot}{\textbf{SHOULD NOT}\xspace}
% MAY: an item is truly optional
\newcommand{\may}{\textbf{MAY}\xspace}

% command to indicate where certain information should be reported
\newcommand{\paper}{PAPER\xspace}
\newcommand{\supplementarymaterial}{SUPPLEMENTARY MATERIAL\xspace}

% configure enumerate/itemize
\usepackage[inline]{enumitem}

% commands to reference sections

% scope
\newcommand{\scope}{\href{/scope/}{Motivation and Scope}\xspace}

% study types
\newcommand{\studytypes}{\href{/study-types}{Study Types}\xspace}
\newcommand{\llmsforresearcher}{\href{/study-types/\#introduction-llms-as-tools-for-software-engineering-researchers}{LLMs as Tools for Software Engineering Researchers}\xspace}
\newcommand{\annotators}{\href{/study-types\#llms-as-annotators}{LLMs as Annotators}\xspace}
\newcommand{\judges}{\href{/study-types\#llms-as-judges}{LLMs as Judges}\xspace}
\newcommand{\synthesis}{\href{/study-types\#llms-for-synthesis}{LLMs for Synthesis}\xspace}
\newcommand{\subjects}{\href{/study-types\#llms-as-subjects}{LLMs as Subjects}\xspace}
\newcommand{\llmsforengineers}{\href{/study-types/\#introduction-llms-as-tools-for-software-engineers}{LLMs as Tools for Software Engineers}\xspace}
\newcommand{\llmusage}{\href{/study-types\#studying-llm-usage-in-software-engineering}{Studying LLM Usage in Software Engineering}\xspace}
\newcommand{\newtools}{\href{/study-types\#llms-for-new-software-engineering-tools}{LLMs for New Software Engineering Tools}\xspace}
\newcommand{\benchmarkingtasks}{\href{/study-types\#benchmarking-llms-for-software-engineering-tasks}{Benchmarking LLMs for Software Engineering Tasks}\xspace}

%guidelines
\newcommand{\guidelines}{\href{/guidelines}{Guidelines}\xspace}
\newcommand{\usagerole}{\href{/guidelines\#declare-llm-usage-and-role}{Declare LLM Usage and Role}\xspace}
\newcommand{\modelversion}{\href{/guidelines\#report-model-version-configuration-and-customizations}{Report Model Version, Configuration, and Customizations}\xspace}
\newcommand{\toolarchitecture}{\href{/guidelines\#report-tool-architecture-beyond-models}{Report Tool Architecture beyond Models}\xspace}
\newcommand{\humanvalidation}{\href{/guidelines\#use-human-validation-for-llm-outputs}{Use Human Validation for LLM Outputs}\xspace}
\newcommand{\prompts}{\href{/guidelines\#report-prompts-their-development-and-interaction-logs}{Report Prompts, their Development, and Interaction Logs}\xspace}
\newcommand{\openllm}{\href{/guidelines\#use-an-open-llm-as-a-baseline}{Use an Open LLM as a Baseline}\xspace}
\newcommand{\benchmarksmetrics}{\href{/guidelines\#use-suitable-baselines-benchmarks-and-metrics}{Use Suitable Baselines, Benchmarks, and Metrics}\xspace}
\newcommand{\limitationsmitigations}{\href{/guidelines\#report-limitations-and-mitigations}{Report Limitations and Mitigations}\xspace}

\begin{document}

\subsection{Use Human Validation for LLM Outputs}

\subsubsection{Recommendations}

\begin{quote}
\underline{tl;dr:} Researchers \must use human validation for LLM outputs when no reference dataset exists or when automated metrics are insufficient. They \must define the measured construct and motivate and describe the validation method in the \paper. Researchers \should plan the human validation from the outset, use validated instruments, build on established reference models for human-LLM comparison, and share the instruments as \supplementarymaterial. When aggregating judgments, methods and rationale \should be reported and inter-rater agreement \should be assessed. Confounding factors \should be controlled for, and power analysis \should be conducted to ensure statistical robustness.
\end{quote}

While LLMs can automate many tasks in research and software development that have traditionally been performed by humans, it is important to validate LLM outputs with human judgment.
For example, in natural language processing tasks, a large-scale study has shown that LLMs have a significant variation in their results, which limits their reliability as a direct substitute for human judges~\cite{DBLP:journals/corr/abs-2406-18403}. 
Moreover, LLMs fail to match human annotators in labeling tasks for natural language inference, stance detection, semantic change, and hate speech detection~\cite{DBLP:conf/chi/Wang0RMM24}.
Human validation helps ensure the accuracy and reliability of the results, as LLMs may sometimes produce incorrect or biased outputs.

Researchers \must employ human validation of LLM outputs whenever there is no established reference dataset or when automated metrics alone cannot fully capture the target qualities they are concerned with.
However, researchers \should use systematic approaches to decide how human annotations can be replaced with LLM-generated ones, such as the methods proposed by Ahmed et al.~\cite{DBLP:journals/corr/abs-2408-05534}.
Their suggested method involves using a jury of three LLMs with 3-4 few-shot examples rated by humans, where the model-to-model agreement on all samples is determined using Krippendorff's Alpha.
If the agreement is high (alpha $gt$ 0.5), a human rating can be replaced with a with an LLM-generated one.
In cases of low model-to-model agreement (alpha $\le$ 0.5), they then evaluate the prediction confidence of the model, selectively replacing annotations where the model confidence is high ($\ge$ 0.8).

Researchers \should plan human validation from the outset and integrate it into their study design.
Integrating human participants in the study design will require additional considerations including a recruitment strategy, annotation guidelines, training sessions, or ethical approvals.
Researchers \should use established reference models for comparing humans with LLMs.
For example, the reference model by Schneider et al.~\cite{Schneider2025ReferenceModel} provides researchers with an overview of the design considerations studies comparing LLMs with humans.
They outline the relevant groups (researchers, experimenters, subjects), activities (study planning, task assignment, results analysis, interpretation), and artifacts (study design, tasks, results, conclusion).
Further, they illustrate the relationship between these entities, the order of steps, and which design decisions need to be considered.

When conducting empirical measurements through human validation, researchers \must clearly define the construct that they are measuring and \must specify the methods used for measurement in their \paper.
Further, they \should use established measurement methods and instruments that are empirically validated~\cite{DBLP:journals/fcomp/HoffmanMKL23, DBLP:conf/chi/PerrigSB23}.
When creating prepared instruments to instruct human participants in validating LLM outputs (e.g., questionnaires, study tasks), researchers \should share the instruments in their \supplementarymaterial.

Validating the output of an LLM \may involve the aggregation of inputs from multiple human judges.
In these cases, researchers \should clearly describe their method of aggregation and document their reasoning for doing so.
When multiple humans are annotating the same artifact, researchers \should validate the agreement between multiple validators with inter-rater reliability measures such as Cohen's Kappa or Krippendorff's Alpha.
When employing human validation, additional confounding factors \should be controlled for through methods such as stratified sampling or by categorizing participants based on experience levels.
Where applicable, researchers \should conduct a power analysis to estimate the required sample size and ensure sufficient statistical power in their experiment design.


\subsubsection{Example(s)}

Ahmed et al.~\cite{DBLP:journals/corr/abs-2408-05534} evaluated the how human annotations can be replaced by LLM generated labels.
In their study, they explicitly report the calculated agreement metrics between the models and between the humans.
For instance, ``Model-model agreement is high, for all criteria, especially for the three large models (GPT-4, Gemini, and Claude). Table I indicates that the mean Krippendorff’s $\alpha$ is 0.68-0.76. 
Second, we see that human-model and human-human agreements are in similar ranges, 0.24-0.40 and 0.21-0.48
for the first three categories.''.

Xue et al.~\cite{DBLP:conf/icse/XueCBTH24} conducted a controlled experiment in which they evaluated the impact of ChatGPT on the performance and perceptions of students in an introductory programming course.
They employed multiple measures to judge the impact of the LLM from the perspective of humans.
In their study, they recorded the students' screens, evaluated the answers they provided in tasks, and distributed a post-study survey to get direct opinions from the students.

Hymel et al.~\cite{hymel2025analysisllmsvshuman} evaluated the capability of ChatGPT-4.0 to generate requirements documents. 
Specifically, they evaluated two requirements documents based on the same business use case, one document generated with the LLM and one document created by a human expert.
The documents were then reviewed by experts and judged in terms of alignment with the original business use case (scale of 1-10), requirements completion (Not Complete, Fairly Complete, Fully Complete), and whether they believed it was created by a human or an LLM.
Finally, they analyzed the influence of the participants' familiarity with AI tools on the study results.
Participants self-reported their AI tool proficiency (Novice, Intermediate, Advanced, Expert) and usage frequency (Daily, Weekly, Monthly, Never), which were then correlated with their ratings of the requirements documents.


\subsubsection{Advantages}

Incorporating human judgment in the evaluation process adds a layer of quality control and increases the trustworthiness of the study's findings, especially when explicitly reporting inter-rater reliability metrics~\cite{khraisha2024canlargelanguagemodelshumans}.

Incorporating feedback from individuals from the target population strengthens external validity by grounding study findings in real-world usage scenarios and may positively impact the transfer of study results to practice.
Researchers may uncover additional opportunities to further improve the LLM or LLM-based tool based on the reported experiences.

\subsubsection{Challenges}

Measurement through human validation can be challenging.
Ensuring that the operationalization of a desired construct and the method of measuring it are appropriate requires a good understanding of the studied concept and construct validity in general, and a systematic design approach for the measurement instruments~\cite{DBLP:journals/tse/SjobergB23}.

Human judgment is subjective, which may lead to variability between different subjects due to differences in expertise, interpretation, and biases among evaluators~\cite{DBLP:journals/pacmhci/McDonaldSF19}.
Controlling for this subjectivity will require additional rigor when analyzing the study results.
Different individuals might have different expectations and experiences that might impact the results of human evaluation. For example, \enquote{the struggle with adapting to AI-assisted work is more common for racially minoritized developers. That group also rated the quality of the output of AI-assisted coding tools significantly lower than other groups}~\cite{hicks_lee_foster-marks_2025}.

Recruiting participants as human validators will always incur additional resources compared to machine-generated measures.
Researchers must weigh the cost and time investment incurred by the recruitment process against the potential benefits for the validity of their study results.

\subsubsection{Study Types}

For \href{/study-types/#studying-llm-usage-in-software-engineering}{Studying LLM Usage in Software Engineering}, researchers \should consider the validity of their evaluation criteria.
In studies where there exist arguably objective evaluation criteria, such as \href{/study-types/#benchmarking-llms-for-software-engineering-tasks}{Benchmarking LLMs for Software Engineering Tasks}, there is little need for human validation, as the benchmark is (hopefully) already validated.

Where criteria are more unclear, such as \href{/study-types/#llms-as-annotators}{LLMs as Annotators}, \href{/study-types/#llms-as-subjects}{LLMs as Subjects}, or \href{/study-types/#llms-for-synthesis}{LLMs for Synthesis}, human validation is important, and these guidelines apply. When using \href{/study-types/#llms-as-judges}{LLMs as Judges}, it is common to have a human first co-create the rating criteria. 

In developing \href{/study-types/#llms-for-new-software-engineering-tools}{LLMs for New Software Engineering Tools}, human validation of the LLM output is critical to ensure the outputs are expected.
Researchers \should make sure to follow the guidelines outlined in this section when employing human validation.

\subsubsection{References}

\bibliographystyle{plainnat}
\bibliography{../../literature.bib}

\end{document}
