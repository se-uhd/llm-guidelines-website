\documentclass[pdftex,11pt,a4paper]{article}

% language and encoding
\usepackage[utf8]{inputenc} % set document encoding to UTF-8
\usepackage[main=english, ngerman]{babel} % adjust language depending on thesis language (affects hyphenation)
\usepackage[T1]{fontenc} % set font encoding so that special characters are displayed correctly

% layout and formatting
\usepackage[paper=a4paper, inner=30mm, outer=25mm, top=30mm, bottom=25mm]{geometry}  % set page margins
\usepackage[parfill]{parskip} % use newlines for paragraphs (more similar to Markdown)
\usepackage{xspace}  % control whitespaces
\usepackage[hyphens]{url} % line breaks in URLs
\usepackage{hyperref}
\usepackage{csquotes}% environment for quotes
\usepackage[super]{nth}
\newcommand*{\enq}[1]{\enquote{{\itshape#1}}} % use italics font for quotes

% math formulas
\usepackage{amsmath}
\usepackage{amssymb} 
\usepackage{eucal} % more curly versions for \mathcal{...}
\usepackage{nicefrac} % nicer fractions
\usepackage{bm} % bold math symbols

% citations
\usepackage[square,numbers,sort&compress]{natbib}

% editorial commands
\newcommand{\todo}[1]{{\textbf{TODO:}\ \textit{#1}}} % command for TODOs
\newcommand{\comment}[1]{{\textbf{Comment:}\ \textit{#1}}} % command for review comments

% RFC 2119 (https://www.rfc-editor.org/rfc/rfc2119)
% MUST: absolute requirement
\newcommand{\must}{\textbf{MUST}\xspace}
% MUST NOT: absolute prohibition
\newcommand{\mustnot}{\textbf{MUST NOT}\xspace}
% SHOULD: there may exist valid reasons in particular circumstances to ignore a  particular item, but the full implications must be understood and carefully weighed before choosing a different course
\newcommand{\should}{\textbf{SHOULD}\xspace}
% SHOULD NOT: there may exist valid reasons in particular circumstances when the particular behavior is acceptable or even useful, but the full implications should be understood and the case carefully weighed before implementing any behavior described with this label
\newcommand{\shouldnot}{\textbf{SHOULD NOT}\xspace}
% MAY: an item is truly optional
\newcommand{\may}{\textbf{MAY}\xspace}

% command to indicate where certain information should be reported
\newcommand{\paper}{PAPER\xspace}
\newcommand{\supplementarymaterial}{SUPPLEMENTARY MATERIAL\xspace}

% configure enumerate/itemize
\usepackage[inline]{enumitem}

% commands to reference sections

% scope
\newcommand{\scope}{\href{/scope/}{Motivation and Scope}\xspace}

% study types
\newcommand{\studytypes}{\href{/study-types}{Study Types}\xspace}
\newcommand{\llmsforresearcher}{\href{/study-types/\#introduction-llms-as-tools-for-software-engineering-researchers}{LLMs as Tools for Software Engineering Researchers}\xspace}
\newcommand{\annotators}{\href{/study-types\#llms-as-annotators}{LLMs as Annotators}\xspace}
\newcommand{\judges}{\href{/study-types\#llms-as-judges}{LLMs as Judges}\xspace}
\newcommand{\synthesis}{\href{/study-types\#llms-for-synthesis}{LLMs for Synthesis}\xspace}
\newcommand{\subjects}{\href{/study-types\#llms-as-subjects}{LLMs as Subjects}\xspace}
\newcommand{\llmsforengineers}{\href{/study-types/\#introduction-llms-as-tools-for-software-engineers}{LLMs as Tools for Software Engineers}\xspace}
\newcommand{\llmusage}{\href{/study-types\#studying-llm-usage-in-software-engineering}{Studying LLM Usage in Software Engineering}\xspace}
\newcommand{\newtools}{\href{/study-types\#llms-for-new-software-engineering-tools}{LLMs for New Software Engineering Tools}\xspace}
\newcommand{\benchmarkingtasks}{\href{/study-types\#benchmarking-llms-for-software-engineering-tasks}{Benchmarking LLMs for Software Engineering Tasks}\xspace}

%guidelines
\newcommand{\guidelines}{\href{/guidelines}{Guidelines}\xspace}
\newcommand{\usagerole}{\href{/guidelines\#declare-llm-usage-and-role}{Declare LLM Usage and Role}\xspace}
\newcommand{\modelversion}{\href{/guidelines\#report-model-version-configuration-and-customizations}{Report Model Version, Configuration, and Customizations}\xspace}
\newcommand{\toolarchitecture}{\href{/guidelines\#report-tool-architecture-beyond-models}{Report Tool Architecture beyond Models}\xspace}
\newcommand{\humanvalidation}{\href{/guidelines\#use-human-validation-for-llm-outputs}{Use Human Validation for LLM Outputs}\xspace}
\newcommand{\prompts}{\href{/guidelines\#report-prompts-their-development-and-interaction-logs}{Report Prompts, their Development, and Interaction Logs}\xspace}
\newcommand{\openllm}{\href{/guidelines\#use-an-open-llm-as-a-baseline}{Use an Open LLM as a Baseline}\xspace}
\newcommand{\benchmarksmetrics}{\href{/guidelines\#use-suitable-baselines-benchmarks-and-metrics}{Use Suitable Baselines, Benchmarks, and Metrics}\xspace}
\newcommand{\limitationsmitigations}{\href{/guidelines\#report-limitations-and-mitigations}{Report Limitations and Mitigations}\xspace}

% custom format for subsubsections and paragraphs
\newcommand{\guidelinesubsubsection}[1]{\subsubsection{#1}}
\newcommand{\studytypesubsection}[1]{\subsection{#1}}
\newcommand{\studytypeparagraph}[1]{\subsubsection{#1}}
\newcommand{\scopeparagraph}[1]{\subsubsection{#1}}

% adjust layout depending on document type
\newcommand{\modelversionsummary}{In summary, our recommendation is to report:

\textbf{(1)} Model/tool name and version (\must in \paper);
\textbf{(2)} All relevant configured parameters that affect output generation (\must in \paper);
\textbf{(3)} Default values of all available parameters (\should);
\textbf{(4)} Checksum/fingerprint of used model version and configuration (\may);
\textbf{(5)} Additional properties such as context window size (\may).

For fine-tuned models, additional recommendations apply:

\textbf{(1)} Fine-tuning goal (\must in \paper);
\textbf{(2)} Fine-tuning dataset creation and characterization (\must in \paper);
\textbf{(3)} Fine-tuning parameters and procedure (\must in \paper);
\textbf{(4)} Fine-tuning dataset and fine-tuned model weights (\should);
\textbf{(5)} Validation metrics and benchmarks (\should).
}
\newcommand{\judgesexample}{\begin{quote}
\small
Your task is to evaluate the quality of a software requirement.\\
Evaluate whether the following requirement is \{quality\_characteristic\}. \\
\{quality\_ characteristic\} means: \{quality\_characteristic\_explanation\}\\
The evaluation result must be: `yes' or `no'.\\
Request: Based on the following description of the project:
\{project\_description\}\\
Evaluate the quality of the following requirement: \{requirement\}.\\
Explain your decision and suggest an improved version.\\
\end{quote}
}


\begin{document}

\subsection{Report Limitations and Mitigations}
\label{sec:report-limitations-and-mitigations}

\begin{quote}
\underline{tl;dr:} Researchers \must transparently report study limitations, including the impact of non-determinism and generalizability constraints. The \paper~\must specify whether generalization across LLMs or across time was assessed, and discuss model and version differences. Authors \must describe measurement constructs and methods, disclose any data leakage risks, and avoid leaking evaluation data into LLM improvement pipelines. They \must provide model outputs, discuss sensitive data handling, ethics approvals, and justify LLM usage in light of its resource demands.
\end{quote}


When using LLMs for empirical studies in SE, researchers face unique challenges and potential limitations that can influence the validity, reliability, and reproducibility of their findings~\cite{sallou2024breaking}.
It is important to openly discuss these limitations and explain how their impact was mitigated.
All this is relative to the current capabilities of LLMs and the current state of the art in terms of tool architectures.
If and how the performance of LLMs in a particular study context will improve with future model generations or new architectural patterns is beyond what researchers can and should discuss as part of a paper's limitation section.
Nevertheless, risk management and threat mitigation are important tasks during the design of an empirical study.
They should not happen as an afterthought.

\subsubsection{Recommendations}

A cornerstone of open science is the ability to reproduce research results.
Although the inherent non-deterministic nature of LLM is a strength in many use cases, its impact on \emph{reproducibility} is a challenge.
To enable reproducibility, researchers \should disclose a replication package for their study.
They \should perform multiple repetitions of their experiments (see \benchmarksmetrics) to account for non-deterministic outputs.
In some cases, researchers \may reduce the output variability by setting the temperature to a value close to 0 and setting a fixed seed value.
However, configuring a lower temperature can negatively impact task performance, and not all models allow the configuration of seed values.
Besides the inherent non-determinism, the behavior of an LLM depends on many external factors such as prompt variations and model evolution.
To ensure reproducibility, researchers \should follow all previous guidelines and further discuss the generalization challenges outlined below.

Although the topic of \emph{generalizability} is not new, it has gained new relevance with increasing interest in LLMs.
In LLM-based studies, generalizability boils down to two main concerns:
(1) First, are the results specific to an LLM, or can they be achieved with other LLMs as well? 
(2) Second, will these results still be valid in the future?
If generalizability to other LLMs is not in the scope of the research, this \must be clearly explained in the \paper or, if generalizability is in scope, researchers \must compare their results or subsets of the results (e.g., due to computational cost) with other LLMs that are, e.g., similar in size, to assess the generalizability of their findings (see also Section \openllm).
Multiple studies (e.g., \cite{DBLP:journals/corr/abs-2307-09009, doi:10.1148/radiol.232411}) found that the performance of proprietary LLMs (e.g., GPT) decreased over time for certain tasks using the same model version (e.g. GPT-4o).
Reporting the model version and configuration is not sufficient in such cases.
To date, the only way to mitigate this limitation is the usage of an open LLM having a versioning schema and being archiving (see \openllm).
Hence, researchers \should employ open LLMs to establish a reproducible baseline and, if the use of an open LLM is not possible, researchers \should test and report their results over an extended period of time as a proxy of the stability of the results over time.

\emph{Data leakage, contamination, or overfitting} occurs when information outside the training data influences model performance. %, leading to overly optimistic performance estimates.
With the growing reliance on big datasets, the risks of inter-dataset duplication increases (see, e.g.,~\cite{DBLP:journals/pacmpl/LopesMMSYZSV17, DBLP:conf/oopsla/Allamanis19}). 
In the context of LLMs for SE, this can manifest itself, for example, as training data samples that appear in the fine-tuning or evaluation datasets, potentially compromising the validity of the evaluation results~\cite{DBLP:journals/tse/LopezCSSV25}.
Moreover, ChatGPT's functionality to ``improve the model for everyone'' can result in unintentional data leakage.
Hence, to ensure the validity of the evaluation results, researchers \should carefully curate the fine-tuning and evaluation datasets to prevent inter-dataset duplication and \mustnot leak their fine-tuned or evaluation datasets into LLM improvement processes.
When publishing the results, researchers \should of course still follow open science practices and publish the datasets as part of their \supplementarymaterial.
If information about the training data of the employed LLM is available, researchers \should evaluate the inter-dataset duplication and \must discuss potential data leakage in the \paper.
When training an LLM from scratch, researchers \may consider using open datasets such as \emph{RedPajama (Together~AI)}~\cite{together2023redpajama}, which are already built with deduplication in mind (with the positive side effect of potentially improving performance~\cite{DBLP:conf/acl/LeeINZECC22}).

Conducting studies with LLMs is a resource-intensive process.
For self-hosted LLMs, the respective hardware must be provided, and for managed LLMs, the service \emph{costs} must be considered.
The challenge becomes more pronounced as LLMs grow larger, research architectures become more complex, and experiments become more computationally expensive.
For example, multiple repetitions to assess model or tool performance (see \benchmarksmetrics) multiply the cost and impact \emph{scalability}.
Consequently, resource-intensive research remains predominantly the domain of private companies or well-funded research institutions, hindering researchers with limited resources in reproducing, replicating, or extending study results.
Hence, for transparency reasons, researchers \should report the cost associated with executing an LLM-based study. 
If the study used self-hosted LLMs, researchers \should report the specific hardware used. 
If the study used managed LLMs, the service cost \should be reported.
To ensure the validity and reproducibility of the research results, researchers \must provide the LLM outputs as evidence (see Section \prompts).
Depending on the architecture (see Section \toolarchitecture), these outputs \should be reported on different architectural levels (e.g., outputs of individual LLMs in multi-agent systems).
Additionally, researchers \should include a subset of the employed validation dataset, selected using an accepted sampling strategy, to allow partial replication of the results.

\emph{Sensitive data} can range from personal to proprietary data, each with its own set of \emph{ethical concerns}.
As mentioned above, a big threat to proprietary LLMs and sensitive data is its use for model improvement.
Hence, using sensitive data can lead to privacy and intellectual property (IP) violations.
Another threat is the implicit bias of LLMs that could lead to discrimination or unfair treatment of individuals or groups.
To mitigate these concerns, researchers \should minimize the sensitive data used in their studies, \must follow applicable regulations (e.g., GDPR) and individual processing agreements, \should create a data management plan outlining how the data is handled and protected against leakage and discrimination, and \must apply for approval from the ethics committee of their organization (if required).

%\enquote{The field of AI is currently primarily driven by research that seeks to maximize model accuracy — progress is often used synonymously with improved prediction quality. This endless pursuit of higher accuracy over the decade of AI research has significant implications for computational resource requirements and environmental footprint. To develop AI technologies responsibly, we must achieve competitive model accuracy at a fixed or even reduced computational and environmental cost.}~\cite{DBLP:conf/mlsys/WuRGAAMCBHBGGOM22}

Although \emph{metrics} such as \emph{BLEU} or \emph{ROUGE} are commonly used to evaluate the performance of LLMs (see Section \benchmarksmetrics), they may not capture other relevant SE-specific aspects such as functional correctness or runtime performance of automatically generated source code~\cite{DBLP:conf/nips/LiuXW023}.
%Researchers \should clarify and state all relevant requirements, and employ and report the metrics to measure the satisfaction of the requirements (e.g., test-case success rate) and \should follow the best practices described in \href{/guidelines/report-baselines-benchmarks-and-metrics}{Report Suitable Baselines, Benchmarks, and Metrics}.
Given the high resource demand of LLMs, in addition to traditional metrics such as accuracy, precision, and recall or more contemporary metrics such as \emph{pass@k}, \emph{resource consumption} has to become a key indicator of performance. % to assess research progress responsibly. 
While research has predominantly focused on energy consumption during the early phases of building LLMs (e.g., data center manufacturing, data acquisition, training), inference, that is LLM usage, often becomes similarly or even more resource-intensive~\cite{de2023growing, DBLP:conf/mlsys/WuRGAAMCBHBGGOM22, DBLP:journals/corr/abs-2410-02950, JIANG2024202, mitu2024hidden}.
Hence, researchers \should aim for lower resource consumption on the model side.
This can be achieved by selecting smaller (e.g., GPT-4o-mini instead of GPT-4o) or newer models or by employing techniques such as model pruning, quantization, or knowledge distillation~\cite{mitu2024hidden}.
Researchers \may further reduce resource consumption by restricting the number of queries, input tokens, or output tokens~\cite{mitu2024hidden}, by using different prompt engineering techniques (e.g., zero-shot prompts seem to emit less CO2 than chain-of-thought prompts), or by carefully sampling smaller datasets for fine-tuning and evaluation instead of using large datasets.
Reducing resource consumption involves a trade-off with our recommendation to perform multiple experimental runs to account for non-determinism, as suggested in \benchmarksmetrics.
To report the environmental impact of a study, researchers \should use software such as \href{https://github.com/mlco2/codecarbon}{CodeCarbon} or \href{experiment-impact-tracker}{Experiment Impact Tracker} to track and quantify the carbon footprint of the study or report an estimate of the carbon footprint through tools such as \href{https://mlco2.github.io/impact/#about}{MLCO2 Impact}.
They \should detail the LLM version and configuration as described in \modelversion, state the hardware or hosting provider of the model as described in \toolarchitecture and report the total number of requests, accumulated input and output tokens.
Researchers \must justify why LLMs were chosen over existing approaches and discuss the achieved performance in relation to their potentially higher resource consumption.

\subsubsection{Example(s)}

An example highlighting the need for caution around \emph{replicability} is the study of \citeauthor{DBLP:conf/sigir-ap/StaudingerKPLH24}~\cite{DBLP:conf/sigir-ap/StaudingerKPLH24} who attempted to replicate an LLM study.
The authors were unable to reproduce the exact results, even though they saw similar trends as in the original study.
%They consider their results as not reliable enough for a systematic review.

To analyze whether the results of proprietary LLMs transfer to open LLMs, \citeauthor{DBLP:conf/sigir-ap/StaudingerKPLH24} benchmarked previous results using GPT-3.5 and GPT4 against Mistral and Zephyr~\cite{DBLP:conf/sigir-ap/StaudingerKPLH24}.
They found that open-source models could not deliver the same performance as proprietary models. %, restricting the effect to certain proprietary models.
This paper is also an example of a study reporting \emph{costs}: \enquote{120 USD in API calls for GPT 3.5 and GPT 4, and 30 USD in API calls for Mistral AI. Thus, the total LLM cost of our reproducibility study was 150 USD}.
\citeauthor{tinnessoftware} is an example of balancing dataset size between the need for manual semantic analysis and computational resource consumption~\cite{tinnessoftware}.

Individual studies have already begun to highlight the uncertainty about the \emph{generalizability} of their results in the future. \citeauthor{DBLP:conf/msr/JesseADM23} acknowledge the issue that LLMs evolve over time and that this evolution could impact the study results~\cite{DBLP:conf/msr/JesseADM23}.
Since a lot of research in SE evolves around code, inter-dataset code duplication has been extensively researched over the years to curate de-duplicated datasets (see, e.g., \cite{DBLP:journals/pacmpl/LopesMMSYZSV17, DBLP:conf/oopsla/Allamanis19, DBLP:journals/ese/KarmakarAR23, DBLP:journals/tse/LopezCSSV25}).
The issue of inter-dataset duplication has also attracted interest in other disciplines. %, with growing demands for data mining.
For example, in biology, \citeauthor{DBLP:journals/biodb/LakiotakiVTGT18} acknowledge and address the overlap between multiple common disease datasets~\cite{DBLP:journals/biodb/LakiotakiVTGT18}. 
In the domain of code generation, \citeauthor{DBLP:conf/ease/CoignionQR24} evaluated the performance of LLMs to produce leet code~\cite{DBLP:conf/ease/CoignionQR24}.
To mitigate the issue of inter-dataset duplication, they only used leet code problems published after 2023-01-01, reducing the likelihood of LLMs having seen those problems before.
Further, they discuss the performance differences of LLMs on different datasets in light of potential inter-dataset duplication.
\citeauthor{zhou2025lessleakbenchinvestigationdataleakage} performed an empirical evaluation of data leakage in 83 software engineering benchmarks~\cite{zhou2025lessleakbenchinvestigationdataleakage}.
Although most benchmarks suffer from minimal leakage, very few showed a leakage of up to 100\%.
The authors found a high impact of data leakage on the performance evaluation.
A starting point for studies that aim to assess and mitigate inter-dataset duplication are the \href{https://huggingface.co/datasets/tiiuae/falcon-refinedweb}{Falcon LLMs}, for which parts of its training data are available on Hugging Face~\cite{technology_innovation_institute_2023}.
Through this dataset, it is possible to reduce the overlap between the traning and evaluation data, improving the validity of the evaluation results.
A starting point to prevent active data leakage into a LLM improvement process is to ensure that the data is not used to train the model (e.g., via OpenAI's data control functionality)~\cite{DBLP:conf/eacl/BalloccuSLD24}.

Bias can occur in LLM training datasets, resulting in various types of discrimination.
\citeauthor{DBLP:journals/corr/abs-2309-00770} propose metrics to quantify biases in various tasks (e.g., text generation, classification, question answering)~\cite{DBLP:journals/corr/abs-2309-00770}.

\subsubsection{Advantages}

%Ensuring \emph{reproducibility} allows for the independent replication and verification of study results.
The \emph{reproduction} or \emph{replication} of study results under similar conditions by different parties greatly increases the validity of the results. % and promotes research transparency.
Independent verification is of particular importance for studies involving LLMs, due to the non-determinism of their outputs and the potential for biases in their training, fine-tuning, and evaluation datasets.
Mitigating threats to \emph{generalizability} of a study through the integration of an open LLM as a baseline or the reporting of results over an extended period of time can increase the validity, reliability, and replicability of a study's results.
Assessing and mitigating the effects of \emph{inter-dataset duplication} strengthens a study's validity and reliability, as it prevents overly optimistic performance estimates that do not apply to previously unknown samples.
Reporting the \emph{costs} of performing a study not only increases transparency but also provides context for secondary literature. % in setting primary research into perspective.
Providing \emph{replication packages} with LLM output and data samples for partial replicability are paramount steps toward open and inclusive research in light of resource inequality between researchers.
Considering and justifying the usage of LLMs over other approaches can lead to more efficient and sustainable solutions for SE problems. 
Reporting the \emph{environmental impact} of LLM usage also sets the stage for more sustainable research practices in SE.

\subsubsection{Challenges}

With commercial LLMs evolving over time, the \emph{generalizability} of results to future model versions is uncertain.
Employing open LLMs as a baseline can mitigate this limitation, but may not always be feasible due to computational cost.
Most LLM providers do not publicly offer information about their training data, impeding the assessment of \emph{inter-dataset duplication} effects.
Consistently keeping track of and reporting the \emph{costs} involved in a research endeavor is challenging.
Building a coherent \emph{replication package} that includes LLM outputs and samples for partial replicability requires additional effort and resources.
Finding the right \emph{metrics} to evaluate the performance of LLMs in SE for specific tasks is challenging.
Defining all requirements beforehand to ensure the use of suitable metrics can be challenging, especially in exploratory research.
Our Section \benchmarksmetrics and its references can serve as a starting point.
Ensuring \emph{compliance} across jurisdictions is difficult with different regions having different regulations and requirements (e.g., GDPR and the AI Act in the EU, CCPA in California).
Selecting datasets and models with fewer \emph{biases} is challenging, as bias are often unknown.
Measuring or estimating the \emph{environmental impact} of a study is challenging and might not always be feasible.
Especially in exploratory research, the impact is hard to estimate in advance, making it difficult to justify the usage of LLMs over other approaches.

\subsubsection{Study Types}

Limitations and mitigations \should be discussed for all study types in relation to the context of the individual study.
This section provides a starting point and lists aspects along which researchers can reflect on the limitations of their study design and potential mitigations that exist.

\subsubsection{References}

\bibliographystyle{plainnat}
\bibliography{../../literature.bib}

\end{document}
